% Created 2024-09-25 Wed 00:53
% Intended LaTeX compiler: xelatex
\documentclass[letterpaper]{article}
\usepackage{graphicx}
\usepackage{longtable}
\usepackage{wrapfig}
\usepackage{rotating}
\usepackage[normalem]{ulem}
\usepackage{capt-of}
\usepackage{hyperref}
\usepackage{fontspec}
\usepackage{bookmark}
\setmainfont[Ligatures=TeX]{CMU Serif}
\setlength{\parindent}{0pt}
\author{Gleb Anohin}
\date{\today}
\title{analysis expression}
\hypersetup{
 pdfauthor={Gleb Anohin},
 pdftitle={analysis expression},
 pdfkeywords={},
 pdfsubject={},
 pdfcreator={Emacs 29.1 (Org mode 9.8)}, 
 pdflang={English}}
\begin{document}

\maketitle
\tableofcontents

\section{Выссказывание}
\label{sec:orgb00e563}
\subsection{Определение}
\label{sec:orgb6f77d0}
Повествовательное предложение, которое можно отнести к верным или неверным утверждениям.

Обозначение - a, b, c
\subsection{Примеры}
\label{sec:orge9f71ba}
\begin{itemize}
\item Эта ручка синяя (правдивое выссказывание)
\item Все студента специалитета -- кролики (ложное выссказывание)
\item Слон большой (не выссказывание, потому что не понятно насколько большой и какой слон)
\end{itemize}
\section{Предикат}
\label{sec:org2715b35}
\subsection{Определение}
\label{sec:org0cf3c4b}
Некоторое выражение (не обязательно математическое), зависящее от переменных, такое, что при подстановке в переменные некоторых значений, выражение становится выссказыванием.

Множество допустимых (например по условию задачи) значений -- область определения

Обозначение - A(x, y, \ldots{}), B(z, q, \ldots{})
\subsection{Примеры}
\label{sec:org834a567}
\begin{itemize}
\item ``\(a > 0\)'', \(a \in Z\)
\item ``x - Вася'', x - человек
\item ``Все натуральные числа, заканчивающиеся на 0,2,4,6,8 - четные'' - не предикат, потому что нет аргументов
\end{itemize}
\section{Операции с выссказыванием и предикатами}
\label{sec:orgd2d415c}
\subsection{Отрицание}
\label{sec:orgfd7c3f4}
\subsubsection{Определение}
\label{sec:orgcdfe7b8}
Операция инвертирующая истинность выссказывания.

\textbf{В предложениях обычно применяется к глаголам/свойствам}
\subsubsection{Примеры}
\label{sec:orgf209da0}
\begin{itemize}
\item Доска на полу \(\rightarrow\) Доска не на полу

Доска - объект
На полу - свойство

\item Все студенты родились в 2008. \(\rightarrow\) Не все студенты родились в 2008.

\emph{Что? Мы же говорили, что ``не'' применяется к свойствам?}

В этом случае мы отрицаем ``не все студенты'', чтобы сохранить смысл. Нужно просто думать

A = ``x родился в 2008 году''

\(a = [\forall s \in M: A(s)]\)
\(\neg{a} = [\exists s \in M: \neg{A(s)}]\)
\end{itemize}
\subsection{Конъюнкция}
\label{sec:orgedda6c2}
Логическое И (\(\land\))
\subsection{Дизъюнкция}
\label{sec:org279f8ca}
Логическое ИЛИ (\(\lor\))
\subsection{Импликация}
\label{sec:org0697e21}
Если А то Б (\(\Rightarrow\))

\begin{center}
\begin{tabular}{rrr}
A & B & A \(\rightarrow\) B\\
\hline
0 & 0 & 1\\
0 & 1 & 1\\
1 & 0 & 0\\
1 & 1 & 1\\
\end{tabular}
\end{center}
\section{Теорема}
\label{sec:orgc2531f1}
Пусть a, b, c - выссказывания, тогда

\begin{enumerate}
\item \((a \land b) \land c = a \land (b \land c)\) -- сочетательный закон конъюнкции
\item \((a \lor b) \lor c = a \lor (b \lor c)\) -- сочетательный закон дизъюнкции
\item \(a \land (c \lor b) = (a \land b) \lor (a \land c)\) -- распределительный закон конъюнкции относительно дизъюнкции
\item \(a \lor (c \land b) = (a \lor b) \land (a \lor b)\) --
\item \(a \land b = b \land a\) -- переместительный закон конъюнкции
\item \(a \lor b = b \lor a\) -- переместительный закон дизъюнкции
\item \(a \land (\neg{a}) = T\) -- закон исключения третьего
\item \(\neg{(\neg{a})} = a\)
\item \(\neg{(a \land b)} = \neg{a} \lor \neg{b}\)
\item \(\neg{(a \lor b)} = \neg{a} \land \neg{b}\)
\item \(a \rightarrow b = \neg{a} \lor b\)
\item \(\neg{a \rightarrow b} = {a} \land \neg{b}\)
\item \(a \rightarrow b = \neg{a} \rightarrow \neg{b}\)
\end{enumerate}
\end{document}
