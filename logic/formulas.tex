% Created 2024-09-24 Tue 14:07
% Intended LaTeX compiler: xelatex
\documentclass[letterpaper]{article}
\usepackage{graphicx}
\usepackage{longtable}
\usepackage{wrapfig}
\usepackage{rotating}
\usepackage[normalem]{ulem}
\usepackage{capt-of}
\usepackage{hyperref}
\usepackage{fontspec}
\setmainfont[Ligatures=TeX]{CMU Serif}
\usepackage{bookmark}
\usepackage{amssymb}
\usepackage{amsmath}
\setlength{\parindent}{0pt}
\author{Gleb Anohin}
\date{\textit{{[}2024-09-18 Wed 16:58]}}
\title{logic formulas}
\hypersetup{
 pdfauthor={Gleb Anohin},
 pdftitle={logic formulas},
 pdfkeywords={},
 pdfsubject={Заметки к  лекции},
 pdfcreator={Emacs 29.1 (Org mode 9.8)}, 
 pdflang={English}}
\begin{document}

\maketitle
\tableofcontents

\section{{\bfseries\sffamily TODO} from galery}
\label{sec:org8a43a05}
\section{что-то}
\label{sec:orgb8333d0}
\(F \equiv H \iff \phi(F(x_1, x_2, \ldots, x_n)) = \phi(H(x_1, x_2, \ldots, x_n))\)\\

Две формулы F и H тогда и только тогда, когда формула \(F \iff H\) есть тавтология.\\

Следовательно\\
отношение равносильности между формулами есть отношение эквивалентности.\\
\subsection{Основные равносильности}
\label{sec:org382a9ff}
\(A \leftrightarrow B \equiv (A \rightarrow B) \land (B \rightarrow A) \equiv (A \land B) \lor (\overline{A} \land \overline{B})\)\\
\(A \rightarrow B \equiv \overline{A} \lor B \equiv \overline{A \land \overline{B}}\)\\
\(A \rightarrow B \equiv \overline{B} \rightarrow \overline{A}\)\\
\(A \land B \equiv \overline{A \rightarrow \overline{B}} \equiv \overline{\overline{A}\lor \overline{B}}\)\\
\subsection{Равносильные}
\label{sec:orgfdc0f03}
let \(A \equiv B\)\\

\(\overline{A} \equiv \overline{B}\)\\
\(A \land B \equiv B \land C\)\\
\(A \lor B \equiv B \lor C\)\\

\(A \rightarrow C \equiv B \rightarrow C\)\\
\(C \rightarrow A = C \rightarrow B\)\\
\(A \leftrightarrow C = B \leftrightarrow C\)\\
\subsection{Теорема о подстановках}
\label{sec:org798e030}
\(C_A\) - формула, содержащая в качестве своей подформулы формулу \(A\).\\
Пусть \(C_B\) получается из \(C_A\) заменой формулы \(A\)\\
в этом вхождении на \(B\). Тогда, если \(A \equiv B\), то \(C_A \equiv C_B\)\\

\(C\) - формула, содержащая в которой выделено\\
одно вхождение переменной \(V\).\\
Пусть \(C_X, C_Y\) получается из \(C\) заменой переменной \(V\) на \(X, Y\)\\
соответственно. Тогда, если \(X \equiv Y\), то \(C_X \equiv C_Y\)\\

Всякую формулу алгебры можно заменить равносильной ей формулой, содержащей только логический операции конъюнкции, дизъюнкции и отрицания.\\
\subsection{Штрих Шеффера}
\label{sec:orgd92af91}
\(x \uparrow y = \overline{x \land y}\)\\
\subsection{Стрелка Пирса}
\label{sec:orgc862ce8}
\(x \downarrow y \equiv \overline{x \lor b}\)\\
\section{Двойственность}
\label{sec:org286e1aa}
Символы \(\land, \lor\) называются двойственными друг другу.\\

Формула называется двойственной другой, если все операции заменили на двойственные.\\

Двойственный список - инвертированный по значениям список.\\
\section{фф}
\label{sec:orgf2fd83f}
Пусть \(A(x_i)\) - формула, \(<s_i>\) - список истинности её переменных.\\
Тогда \(A(s_i) \equiv T \iff A^* \equiv F\) на списке \(<t_i>\) двойственном к \(<s_i>\)\\

Приницп двойственности: \(A \equiv B \Rightarrow A^* \equiv B^*\).\\

Можно использовать для нахождения новых равносильностей.\\

\(X \land (Y \lor Z) \equiv (X \land Y) \lor (Z \land Z) \Rightarrow X \lor (Y \land Z) \equiv (X \lor Y) \land (Z \lor Z)\)\\
\section{Пусть}
\label{sec:org8901531}
\(x_1, x_2, \ldots, x_n\) - элементарные высказывания\\

Формула алгебры логики - функция входящих в нее элементарных выссказываний\\

\(A(x_1, x_2, \ldots, x_n)\) - формула составленная из \(x_1, \ldots, x_n\)\\

\(x_1, \ldots x_n\) будем называть логическими переменными этой формулы\\

Конъюктивным одночленом называется конъюкция переменных или их отрицаний. Дизъюктивный - аналогично.\\

ДНФ (дизъюктивная нормальная форма) формулы A называется равносильная ей формула, представляющая собой дизъюнкцию элементарных конъюкций.\\

КНФ (конъюктивная нормальная форма) - наоборот.\\

Всякая формула обладает обеими ДНФ и КНФ.\\
\subsection{Пример}
\label{sec:org6863180}
\((A \uparrow B) \rightarrow \overline{(\overline{C} \rightarrow (B \lor C))}\)\\
\section{Совершенность}
\label{sec:orgc35284b}
ДНФ/КНФ называется совершенной, если\\
\begin{enumerate}
\item в каждую из элементарных дизъюнкций/конъюкций логическая переменная входит только один раз.\\
\item если логическая переменная входит в одну из элементарных дизъюнкций/конъюкций, то она входит и во все остальные.\\
\item Все элементарные дизъюнкции/конъюкции различны.\\
\end{enumerate}

\((x \land y \land z) \lor (\overline{x} \land y \land z) \lor (x \land \overline{y} \land z)\) - СДНФ\\

Каждая не являющаяся тождественно ложной или истинной формула от n аргументов имеет единственную СДНФ/СКНФ.\\
\subsection{{\bfseries\sffamily IDEA} доказать}
\label{sec:orgb854f47}
\section{Приведение к СНФ:}
\label{sec:orgef2f51b}
\subsection{Тождественные преобразования}
\label{sec:org0e2e186}
\subsubsection{Алгоритм}
\label{sec:org23a081f}
\begin{enumerate}
\item СКНФ
\label{sec:orgc379bf9}
\begin{enumerate}
\item Найти КНФ\\
\item Путем добавления единичных противоречий \((a \land \neg{a})\) для недостающих переменных в отдельные слагаемые привести к СКНФ\\
\end{enumerate}
\item СДНФ
\label{sec:org326773d}
\begin{enumerate}
\item Найти ДНФ\\
\item Путем добавления единичных тавтологий \((a \lor \neg{a})\) для недостающих переменных в отдельные множежители привести к СДНФ\\
\end{enumerate}
\end{enumerate}
\subsection{Таблица истинности}
\label{sec:orgde897a5}
\subsubsection{Алгоритм}
\label{sec:org82d117a}
\begin{enumerate}
\item СКНФ
\label{sec:org5de8358}
\begin{enumerate}
\item Нужно выбрать все те значения переменных, когда формула ложна.\\
\item Для каждого набора выписываем элементарную дизъюнкцию. Переменная входит в нее сама если в этом наборе F иначе ее отрицание.\\
\item Образуем конъюнкцию всех дизъюнкций\\
\end{enumerate}
\item СДНФ
\label{sec:orgae02e5d}
\begin{enumerate}
\item Нужно выбрать все те значения переменных, когда формула истинна.\\
\item Для каждого набора выписываем элементарную конъюнкцию. Переменная входит в нее сама если в этом наборе T иначе ее отрицание.\\
\item Образуем дизъюнкцию всех конъюнкций\\
\end{enumerate}
\end{enumerate}
\subsubsection{Пример}
\label{sec:org8125624}
\begin{center}
\begin{tabular}{rrrr}
x & y & z & F(x, y, z)\\
\hline
0 & 0 & 0 & 1\\
0 & 0 & 1 & 0\\
0 & 1 & 0 & 1\\
0 & 1 & 1 & 1\\
1 & 0 & 0 & 1\\
1 & 0 & 1 & 0\\
1 & 1 & 0 & 0\\
1 & 1 & 1 & 1\\
\end{tabular}
\end{center}

СДНФ -- \((\neg{x} \land \neg{y} \land \neg{z}) \lor (\neg{x} \land y \land \neg{z}) \lor \dots\)\\

СКНФ -- \((x \lor y \lor \neg{z}) \land (\neg{x} \lor y \lor \neg{z}) \land (\neg{x} \lor \neg{y} \lor z)\)\\
\section{Проблема разрешимости}
\label{sec:org0b66c27}
\subsection{Критерий тождественной}
\label{sec:org0f10193}

\subsection{\ldots{}}
\label{sec:orga6000c9}
Элементарная дизъюнкция тождественно истин\\
\section{Логическое следование}
\label{sec:orgacb21cd}
\(A \rightarrow B \land B \rightarrow C \Rightarrow A \rightarrow C\)\\

\(H(x_1, \dots, x_n)\) называется логическим следствие формул. \(F_1(x_1, \dots, x_n); \dots; F_m(x_1, \dots, x_n)\) если \(H\) превращается в истинное выссказывания при всякой такой подстановке вместо ее переменных конкретных выссказываний, при которых формулы \(F_1, \dots, F_n\) превращаются в истинные выссказывания.\\

Короче (математически не правильно) \(H \Rightarrow F_1, \dots F_m\) обозначается \(F_1, \dots, F_n \models H\)\\

Из истинности посылок следует истинность вывода.\\

\(F \models H \iff \models F \rightarrow H\) (\(F \rightarrow H\) -- тавтология)\\

\(\forall F_1, \dots, F_m, H : (F_1, \dots F_m \models H) \land (F_1 \land \dots \land F_m \models H) \land (\models (F_1 \land \dots \land \dots F_m ) \rightarrow H)\)\\
\subsection{Свойства}
\label{sec:org50f4c79}
\begin{enumerate}
\item \(F_1, \dots, F_m \models F_i \Rightarrow\) для любого \(i\)\\
\item \(F_1, \dots, f_m \models G_i | j \in 1,\dots,p; G_1, \dots, G_p \models H\), то \(F_1, \dots F_m, \models H\)\\
\end{enumerate}

\(F \equiv H \iff F \models H \land H \models F\)\\

Если формула тавтология, то и ее любое следствие является тавтологией.\\
\(\models F, F \models H \Rightarrow \models H\)\\
\subsection{Правила логических умозаключений}
\label{sec:orgefd55a1}
\((\text{modus ponens}): \frac{F, F \rightarrow G}{G}\)\\
\(\models F, F \rightarrow G \Rightarrow \models G\)\\

\((\text{modus ponens}): \frac{F \rightarrow G, \neg{G}}{\neg{F}}\)\\
\(\models \neg{G}, F \rightarrow G \Rightarrow \models \neg{F}\)\\
\end{document}
